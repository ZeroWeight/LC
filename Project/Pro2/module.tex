\documentclass[UTF8,a4paper]{paper}
\usepackage{ctex}
\usepackage[utf8]{inputenc}
\usepackage{amsmath}
\usepackage{pdfpages}
\usepackage{graphicx}
\usepackage{wrapfig}
\usepackage{listings}
\begin{document}
我们选择牛顿力学对斜坡进行分析来找出他们分不同点,为保持和之前的坐标系一致,我们采用平行于平面和垂直于平面的两个方向作为坐标系建模。假设斜坡倾角为$\alpha$

先对轮子进行受力分析,考虑到轮子是纯滚动状态,取地面顺心作为角速度参考点可以得到
\begin{equation}
(I_w+m_wR^2)\ddot{\theta}=\tau-F_xR-\mu\dot{\theta}+m_w\mathrm{g}R\mathrm{sin}\alpha
\end{equation}

式中考虑到了$x=R\theta$的纯滚条件,同时认为轮胎的摩擦系数满足纯滚条件

下一步研究杆的运动状态,我们取杆和车的连接处作为角动量参考点,这个点所对应的参考系并非是惯性系,非惯性力有力矩作用,表达式为

\begin{equation}
(I_B+m_Bl^2)\ddot{\phi}=m_B\mathrm{g}l\mathrm{sin}(\phi+\alpha)-m_BRl\mathrm{cos}\phi\ddot{\theta}-\tau
\end{equation}

下面考虑杆的平动,可以得到

\begin{equation}
F_x=m_B\ddot{x}
\end{equation}

同时进一步考虑运动学关联,可以得到

\begin{equation}
x-l\mathrm{sin}\phi=R\theta
\end{equation}

求导得到

\begin{equation}
\ddot{x}-l\mathrm{cos}\phi\ddot{\phi}+l\mathrm{sin}\phi\dot{\phi}^2=R\ddot{\theta}
\end{equation}

综合带入方程可以得到
\begin{equation}
(I_w+m_wR^2+m_BR^2)\ddot{\theta}+m_BRl(\mathrm{cos}\phi\ddot{\phi}-\mathrm{sin}\phi\dot{\phi}^2)+\mu\dot{\theta}=\tau+m_w\mathrm{g}R\mathrm{sin}\alpha
\end{equation}
\begin{equation}
(I_B+m_Bl^2)\ddot{\phi}=m_B\mathrm{g}l\mathrm{sin}(\phi+\alpha)-m_BRl\mathrm{cos}\phi\ddot{\theta}-\tau
\end{equation}

下面分析一下这个方程,首先验证正确性,当$\alpha=0$时退化为之前讨论的方程,即

\begin{equation}
(I_w+m_wR^2+m_BR^2)\ddot{\theta}+m_BRl(\mathrm{cos}\phi\ddot{\phi}-\mathrm{sin}\phi\dot{\phi}^2)+\mu\dot{\theta}=\tau
\end{equation}
\begin{equation}
(I_B+m_Bl^2)\ddot{\phi}=m_B\mathrm{g}l\mathrm{sin}\phi-m_BRl\mathrm{cos}\phi\ddot{\theta}-\tau
\end{equation}

其次讨论线性化条件,首先我们应该看到,如果$\phi=0$的话,系统不可能保持平衡,稍加分析我们就可以知道,这里的趋近于0的量应当是$\phi+\alpha$,即车辆实际上车体始终保持着绝对的竖直状态,因此我们记$\phi^*=\phi+\alpha$来进行下一步的线性化工作

线性化得到

\begin{equation}
(I_w+m_wR^2+m_BR^2)\ddot{\theta}+m_BRl\mathrm{cos}\alpha\ddot{\phi^*}+\mu\dot{\theta}=\tau+m_w\mathrm{g}R\mathrm{sin}\alpha
\end{equation}
\begin{equation}
(I_B+m_Bl^2)\ddot{\phi^*}=m_B\mathrm{g}l\phi^*-m_BRl\mathrm{cos}\alpha\ddot{\theta}-\tau
\end{equation}

和之前得到的线性化模型比较得到他们的差别

\begin{equation}
(I_w+m_wR^2+m_BR^2)\ddot{\theta}+m_BRl\ddot{\phi}+\mu\dot{\theta}=\tau
\label{16}
\end{equation}
\begin{equation}
(I_B+m_Bl^2)\ddot{\phi}=m_B\mathrm{g}l\phi-m_BRl\ddot{\theta}-\tau
\label{17}
\end{equation}

首先,部分常量的值发生了一些变化;其次,在对系统进行控制的过程中会加入了恒定的扰动。也就是说,同样的控制策略,如果能够使得小车在平面上稳定运动,那么经过简单的参数修改之后,小车是可以在斜面上以一个加速度运动的,这和轮轴处可以随意旋转导致车体和斜坡之间解耦有一定的关系。

因此,想要做到小车在斜坡和平面上都比较稳定的运动是有可能的,而且是相对比较容易达到的。其中稳定运动的目标应当是小车车体完全竖直(而不是和平面竖直)

当然,还需要考虑斜面的曲率和倾角不能过于极端导致小车出现了不可控的滑动的情况。
\end{document}
