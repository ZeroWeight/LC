\documentclass[UTF8,a4paper]{ctexart}
\usepackage[utf8]{inputenc}
\usepackage{amsmath}
\usepackage{pdfpages}
\usepackage{graphicx}
\usepackage{wrapfig}
\usepackage{listings}
\title{线控项目建模}
\author{张蔚桐\ 2015011493\ 自55}
\begin{document}
\maketitle
选择牛顿力学进行分析

先对轮子进行受力分析,考虑到轮子是纯滚动状态,得到
\begin{equation}
I_w\ddot{\theta}=\tau-F_wR-\mu\dot{\theta}
\label{1}
\end{equation}
\begin{equation}
m_wR\ddot{\theta}=-F_x+F_w
\label{2}
\end{equation}
式中考虑到了$x=R\theta$的纯滚条件,同时认为轮胎的摩擦系数满足纯滚条件

下一步研究杆的运动状态,角量上有
\begin{equation}
I_B\ddot{\phi}=F_yl\mathrm{sin}\phi-F_xl\mathrm{cos}\phi
\label{3}
\end{equation}
平动方程
\begin{equation}
F_x=m_B\ddot{x}
\label{4}
\end{equation}
\begin{equation}
m_B\mathrm{g}-F_y=m_B\ddot{y}
\label{5}
\end{equation}
运动学关联我们有
\begin{equation}
x-l\mathrm{sin}\phi=R\theta
\label{6}
\end{equation}
\begin{equation}
y=l\mathrm{cos}\phi
\label{7}
\end{equation}
下面开始解方程

从(\ref{1}),(\ref{2})得到
\begin{equation}
I_w\ddot{\theta}+F_xR+m_wR^2\ddot{\theta}=\tau-\mu\dot{\theta}
\label{8}
\end{equation}
结合(\ref{4})得到
\begin{equation}
(I_w+m_wR^2)\ddot{\theta}+m_BR\ddot{x}+\mu\dot{\theta}=\tau
\label{9}
\end{equation}
对(\ref{6})求导得到
\begin{equation}
\ddot{x}-l\mathrm{cos}\phi\ddot{\phi}+l\mathrm{sin}\phi\dot{\phi}^2=R\ddot{\theta}
\label{10}
\end{equation}
带入(\ref{9})得到
\begin{equation}
(I_w+m_wR^2+m_BR^2)\ddot{\theta}+m_BRl(\mathrm{cos}\phi\ddot{\phi}-\mathrm{sin}\phi\dot{\phi}^2)+\mu\dot{\theta}=\tau
\label{11}
\end{equation}
联立(\ref{3}),(\ref{4}),(\ref{5})得到
\begin{equation}
I_B\ddot{\phi}=(m_B\mathrm{g}-m_B\ddot{y})l\mathrm{sin}\phi-m_B\ddot{x}l\mathrm{cos}\phi
\label{12}
\end{equation}
对(\ref{7})式求导得到
\begin{equation}
\ddot{y}=-l\mathrm{cos}\phi\dot{\phi}^2-l\mathrm{sin}\phi\ddot{\phi}
\label{13}
\end{equation}
联立(\ref{10}),(\ref{12}),(\ref{13})得到
\begin{equation}
I_B\ddot{\phi}=(m_B\mathrm{g}+m_Bl(\mathrm{cos}\phi\dot{\phi}^2+\mathrm{sin}\phi\ddot{\phi}))l\mathrm{sin}\phi-m_B(R\ddot{\theta}+l\mathrm{cos}\phi\ddot{\phi}-l\mathrm{sin}\phi\dot{\phi}^2)l\mathrm{cos}\phi
\label{14}
\end{equation}
稍作整理得到
\begin{equation}
I_B\ddot{\phi}=m_B\mathrm{g}l\mathrm{sin}\phi + m_Bl^2\mathrm{sin}(2\phi) \dot{\phi}^2-m_Bl^2 \mathrm{cos}(2\phi)\ddot{\phi}-m_BRl\mathrm{cos}\phi\ddot{\theta}
\label{15}
\end{equation}
联立(\ref{11}),(\ref{15})并进行线性化得到
\begin{equation}
(I_w+m_wR^2+m_BR^2)\ddot{\theta}+m_BRl\ddot{\phi}+\mu\dot{\theta}=\tau
\label{16}
\end{equation}
\begin{equation}
I_B\ddot{\phi}=m_B\mathrm{g}l\phi-m_Bl^2 \ddot{\phi}-m_BRl\ddot{\theta}
\label{17}
\end{equation}
\end{document}