\documentclass[UTF8,a4paper]{ctexart}
\usepackage[utf8]{inputenc}
\usepackage{amsmath}
\usepackage{pdfpages}
\usepackage{graphicx}
\usepackage{wrapfig}
\usepackage{listings}
\title{线控作业1}
\author{张蔚桐\ 2015011493\ 自55}
\begin {document}
\maketitle
\section{}
根据能控性代数判据可以得到$$\mathbf{[B \ AB \ A^2B]}=\begin{pmatrix}
0&0&1 \\
0&1&-6 \\
1&-6&25
\end{pmatrix}$$
$rank(\mathbf{[B\  AB\  A^2B]})=3$,行满秩可得系统能控
\section{}
\subsection{}
显然系统在A为Jordan标准型基础上B矩阵在第二行出现全零行不可控。
\subsection{}
$$\mathbf{[B \ AB \ A^2B]}=\begin{pmatrix}
2&1&3&2&5&4 \\
1&1&2&2&4&4 \\
-1&-1&-2&-2&-4&-4
\end{pmatrix}$$
$rank(\mathbf{[B\  AB\  A^2B]})=2$,行不满秩可得系统不可控
\section{}
$$\mathbf{[B \ AB \ A^2B\ A^3B]}=\begin{pmatrix}
0&1&0&1 \\
1&0&1&0 \\
0&-1&0&-11\\
-1&0&-11&0
\end{pmatrix}$$
$rank(\mathbf{[B \ AB \ A^2B\ A^3B]})=4$,行满秩可得系统能控
$$\mathbf{[C \ CA \ CA^2\ CA^3]^T}=\begin{pmatrix}
1&0&0&0\\
0&1&0&0\\
0&0&-1&0\\
0&0&0&-1\end{pmatrix}$$
$rank(\mathbf{[C \ CA \ CA^2\ CA^3]^T})=4$,列满秩系统可观
\section{}
\subsection{可控条件}
$$\mathbf{[B \ AB ]}=\begin{pmatrix}
1&a+b \\
1&c+d \end{pmatrix}$$
可控条件显然为$a+b \neq c+d$
\subsection{可观条件}
$$\mathbf{[C \ CA ]^T}=\begin{pmatrix}
1&0\\
a&0\end{pmatrix}$$
系统永不可观
\section{}
$$\mathbf{[B \ AB \ A^2B]}=\begin{pmatrix}
0&4&-8\\
4&-4&4\\
3&-6&12
\end{pmatrix}$$
于是$P_1=[0\ 0\ 1]\mathbf{[B \ AB \ A^2B]}^{-1}=\begin{pmatrix}
\frac{1}{4}&-\frac{1}{4}&\frac{1}{3}\end{pmatrix}$
并进一步得到$$\mathbf{T^{-1}}=\begin{pmatrix}
\mathbf{P_1}\\
\mathbf{P_1A}\\
\mathbf{P_1A^2}\end{pmatrix}
=\begin{pmatrix}
\frac{1}{4}&-\frac{1}{4}&\frac{1}{3}\\
-\frac{1}{4}&\frac{1}{2}&-\frac{2}{3}\\
\frac{1}{4}&-\frac{3}{4}&\frac{4}{3}\end{pmatrix}$$
于是有$$\mathbf{A'}=\mathbf{T^{-1}AT}=\begin{pmatrix}
0&1&0\\
0&0&1\\
-2&-5&-4\end{pmatrix}$$
$$\mathbf{B'}=\mathbf{T^{-1}B}=\begin{pmatrix}0\\0\\1\end{pmatrix}$$将系统化为能控标准型
\section{}
系统状态方程可以表示为$$\mathbf{A}=\begin{pmatrix}
0&1&0\\
-3&-4&0\\
2&1&-2\end{pmatrix}\ , \ 
\mathbf{B}=\begin{pmatrix}0\\1\\0\end{pmatrix} \ , \ 
\mathbf{C}=\begin{pmatrix}0&0&1\end{pmatrix}$$ 
$$\mathbf{[B \ AB \ A^2B]}=\begin{pmatrix}
0&1&-4\\
1&-4&13\\
0&1&-4
\end{pmatrix}$$ $rank(\mathbf{[B \ AB \ A^2B]})=2$故系统不可控
$$\mathbf{[C \ CA \ CA^2]}=\begin{pmatrix}
0&0&1\\
2&1&-2\\
7&-4&4
\end{pmatrix}$$ $rank(\mathbf{[C \ CA \ CA^2]})=3$故系统可观

可以得到系统的传递函数为$$\mathbf{G=C(sI-A)^{-1}B}=\frac{1}{s^2+4s+3}$$
\section{}
\subsection{}
$$\mathbf{G=C(sI-A)^{-1}B}=\frac{s+3}{s^2+2s-1}$$
\subsection{}
$$\mathbf{[B \ AB \ A^2B]}=\begin{pmatrix}
0&1&-2\\
0&0&0\\
1&0&1
\end{pmatrix}$$
$rank(\mathbf{[B\  AB\  A^2B]})=2$,行不满秩可得系统不可控,于是构造其能控子系统
取$\mathbf{[B \ AB \ A^2B]}$的前两列作为列向量,构造$$\mathbf{T}=\begin{pmatrix}
0&1&0\\
0&0&1\\
1&0&0\end{pmatrix}$$
于是$$\mathbf{A'=T^{-1}AT}=\begin{pmatrix}
0&1&-4\\
1&-2&2\\
0&0&-2\end{pmatrix}$$
$$\mathbf{b'=T^{-1}b}=\begin{pmatrix}1\\0\\0\end{pmatrix}$$
$$\mathbf{C'=CT}=\begin{pmatrix}1&1&-1\end{pmatrix}$$
得到系统的能控子系统为
$$\Sigma(\begin{pmatrix}
0&1\\
1&-2\end{pmatrix},
\begin{pmatrix}1\\0\end{pmatrix},
\begin{pmatrix}1&1\end{pmatrix})$$
\subsection{}
$$\mathbf{[C \ CA \ CA^2]}=\begin{pmatrix}
1&-1&1\\
-1&0&1\\
3&-6&-1
\end{pmatrix}$$
$rank(\mathbf{[C \ CA \ CA^2]})=3$,系统可观
\section{}
$$\mathbf{G}=\frac{1}{s^2+3s+2}\begin{pmatrix}
2s+3&1\\
3s+6&3s+6\end{pmatrix}$$
于是可以顺利得到$$\mathbf{A}=\begin{pmatrix}
0&1&0&0\\
-2&-3&0&0\\
0&0&0&1\\
0&0&-2&-3\end{pmatrix}\ ,\ \mathbf{B}=\begin{pmatrix}0&0\\1&0\\0&0\\0&1\end{pmatrix}\ ,\ \mathbf{C}=\begin{pmatrix}
3&2&1&0\\
6&3&6&3\end{pmatrix}$$
下面考虑将这个系统抽出能观部分,考察
$$\mathbf{[C \ CA \ CA^2\  CA^3]}=\begin{pmatrix}
3&2&1&0\\
6&3&6&3\\
-4&-3&0&1\\
-6&-3&-6&-3\\
6&5&-2&-3\\
6&3&6&3\\
-10&-9&6&7\\
-6&-3&-6&-3
\end{pmatrix}$$
抽出线性无关列构成$$\mathbf{T}=\begin{pmatrix}
3&2&1&0\\
6&3&6&3\\
1&0&0&0\\
0&1&0&0\end{pmatrix}$$
得到新的能观表述:
$$\mathbf{A=TAT^{-1}}=\begin{pmatrix}
-2&\frac{1}{3}&0&0\\
0&-1&0&0\\
0&0&0&1\\
0&0&-2&-3\end{pmatrix}\ , \ \mathbf{b={Tb}}=\begin{pmatrix}
2&0\\
3&3\\
0&0\\
1&0\end{pmatrix}\ , \ \mathbf{C={CT^{-1}}}==\begin{pmatrix}
1&0&0&0\\
0&1&0&0\end{pmatrix}$$
于是最小实现为
$$\Sigma(\begin{pmatrix}
-2&\frac{1}{3}\\
0&-1\end{pmatrix},
\begin{pmatrix}2&0\\3&3\end{pmatrix},
\begin{pmatrix}1&0\\0&1\end{pmatrix})$$
\section{}
相关处理后有
$$\mathbf{A}=\begin{pmatrix}
-1&2\\
1&0
\end{pmatrix}\ ,\ \mathbf{B}=\begin{pmatrix}-2\\1\end{pmatrix}\ ,\ \mathbf{C}=\begin{pmatrix}1&-1\end{pmatrix}\ , \ \mathbf{d}=1$$
$$\mathbf{[B \ AB ]}=\begin{pmatrix}
-2&4\\
1&-2
\end{pmatrix}$$
$rank(\mathbf{[B\  AB]})=1$,行不满秩可得系统不可控
$$\mathbf{C \ CA ]}=\begin{pmatrix}
1&-1\\
-2&2
\end{pmatrix}$$
$rank(\mathbf{[C\  CA]})=1$,列不满秩可得系统不可观
综上,系统不可观不可控
\end{document}