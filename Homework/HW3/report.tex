\documentclass[UTF8,a4paper]{ctexart}
\usepackage[utf8]{inputenc}
\usepackage{amsmath}
\usepackage{pdfpages}
\usepackage{graphicx}
\usepackage{wrapfig}
\usepackage{listings}
\title{线控作业3}
\author{张蔚桐\ 2015011493\ 自55}
\begin{document}
\maketitle
\section{}
考虑这些二次型的实对称矩阵
$$V_a(x)=\begin{pmatrix}
1& 1 & -1 \\
1& 3 & -2.5\\
-1& -2.5 & 1\\
\end{pmatrix}$$
$V_a$的特征值为-0.74,0.58,5.15。故$V_a$不定
$$V_b(x)=\begin{pmatrix}
-1& 1 & -1 \\
1& -3 & -0.5\\
-1& -0.5 & -11\\
\end{pmatrix}$$
$V_b$的特征值为-11.12,-3.4134,-0.4677。故$V_b$负定
$$V_c(x)=\begin{pmatrix}
1& 2 & 0 \\
2& 5 & 1\\
0& 1 & 1\\
\end{pmatrix}$$
$V_c$的特征值为0,1,6故$V_c$。半正定
\section{}
\subsection{}
构造$V(\mathbf{x})=x_1^2+2x_1x_2+x_2^2$显然$V(\mathbf{x})$正定,且当$\left|\left|\mathbf{x}\right|\right| \rightarrow \infty$时有$V(\mathbf{x}) \rightarrow \infty$
\begin{equation}\begin{aligned}
&V'(\mathbf{x})=\frac{\partial V}{\partial \mathbf{x}}\mathbf{A}\\& =2x_1(-x_1+x_2+x_1^2)+2x_1(-x_2+2x_1-x_1^2)+\\&2x_2(-x_1+x_2+x_1^2)+2x_2(-x_2+2x_1-x_1^2)\\& =2(x_1+x_2)x_1\end{aligned}\end{equation}
\subsection{}
构造$V(\mathbf{x})=x_1^2+x_2^2$显然$V(\mathbf{x})$正定,且当$\left|\left|\mathbf{x}\right|\right| \rightarrow \infty$时有$V(\mathbf{x}) \rightarrow \infty$

\begin{equation}\begin{aligned}
&V'(\mathbf{x})=\frac{\partial V}{\partial \mathbf{x}}\mathbf{A}\\& =2x_1(-x_1+x_1x_2)+2x_2(-x_2-2x_1-x_1^2)\\& =-2x_1^2+-2x_2^2\end{aligned}\end{equation}
显然$V'(\mathbf{x})$负定,系统在原点处大范围渐进稳定
\section{}

\end{document}